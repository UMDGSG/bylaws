\chapter{Impeachment}
\section{Cause}
The Assembly, by a majority vote of its total present and voting membership, may initiate impeachment proceedings for any just cause, including, but not limited to, neglect, unsatisfactory performance, or misrepresentation of duty against appointed or elected officers of any class, branch, or department of the GSG.
\begin{bylaws-number}
  \item The charge of Neglect is defined as failure to perform basic duties as outlined in the GSG Constitution and Bylaws. Criteria by which Executives’ absenteeism may constitute Neglect are dealt with in 1.11.
  \item The charge of Unsatisfactory Performance is defined as any partial performance of basic duties that is judged to be significantly below that of the expected performance.
  \item The charge of Misrepresentation of Duty is defined as actively attempting to mislead the Assembly or the graduate student body, or any part thereof, in matters of GSG governance; and/or serving badly as a Representative according to the judgment of one’s peers.
\end{bylaws-number}

\section{Proceedings}
\begin{bylaws-number}
	\item Upon the initiation of impeachment proceedings, the question of impeachment shall be forwarded to the Governance Committee.
	\item The Governance Committee shall be charged with the unbiased investigation of all matters related to the impeachment charges.
	\begin{bylaws-number}
		\item The Governance Committee shall conclude whether the impeached party should be formally indicted on the charges as brought forth by the Assembly, and shall formulate, for the Assembly, a recommendation to either proceed with or terminate impeachment proceedings.
		\item The Governance Committee shall be charged with the duty of disclosing to the impeached party the timelines and processes involved in any potential impeachment hearing of the committee and the Assembly. This shall include the distribution of these impeachment protocols to the impeached party.
		\item The investigation shall consist of no less than an interview with the impeached party with representation or a witness pending notification of the committee, an interview with the impeaching party, and the gathering of supporting documentation and other such evidence to be presented to the Assembly. If any of the above mentioned parties are unavailable for interview, that information can be taken into account and should be presented to the Assembly.
		\item The committee shall present a copy of its findings and recommendations to the accused party and the impeaching party no less than twenty-four hours before an anticipated presentation of said findings before the Assembly.
	\end{bylaws-number}
	\item Impeachment proceedings shall commence at the next regularly scheduled meeting of the Assembly. By majority vote, the committee may appeal to the Assembly for an extension of the investigation period, not to exceed one additional month or the duration until the next Assembly meeting, whichever is greater. If such an appeal is met favorably, impeachment proceedings shall commence at the next regularly scheduled meeting of the Assembly following the end of the extension period. Such an extension should be avoided unless judged to be absolutely necessary to the work of the committee.
	\begin{bylaws-number}
		\item At said meeting of the Assembly, a designee of the committee shall present the findings of the investigation, with a recommendation to proceed with or terminate impeachment proceedings. This recommendation shall be considered a question for debate and shall be decided by a simple majority of the Assembly’s present and voting membership. The committee’s recommendation shall take the following form:
		“After considering evidence in the form of corroborated personal communications with the involved parties, and any other additional evidence made available to the committee, and after considerable deliberation, it is the decision of this committee to formally recommend that the Assembly [insert recommendation].”
		\begin{enumerate}[i]
			\item Dismiss: Dismiss the [insert member’s official title] of all impeachment charges.
			\item Impeach: Indict the [insert member’s official title] for impeachment under the charge(s) of [insert formal charge(s)].
			\item Dismiss the charge(s) of [insert dismissed charge(s)], and to formally indict the [insert member’s official title] for impeachment under the charge(s) of [insert formal charge(s)].
		\end{enumerate}
		\item Members of the committee wishing to dissent from the majority opinion may present to the Assembly a written dissenting opinion. This dissenting opinion may be forwarded to the Assembly anonymously through the office of the Chair. Anonymous opinions must contain the number of committee members that are signatory.
		\item If the Assembly elects to proceed with the impeachment, the Assembly shall immediately enter executive session and shall conduct an impeachment trial.
		\item The impeachment trial shall be presided over by the Assembly’s Presiding Officer. The individual/s bringing impeachment charges shall serve as the prosecution and the impeached party shall be charged with mounting his/her defense. The impeached party has the right to be aided in his/her defense by counsel. Counsel can be an attorney or not, but must be a member of the Assembly, unless the Assembly, by majority vote, agrees to permit counsel who is not a member to act in this capacity.
		\item At the conclusion of the trial, the Assembly shall engage in debate and vote upon the question of removing the impeached party from office.
		\item The Assembly, by a three-fourths vote of its total present and voting membership, shall remove the impeached party from his/her position(s). If a party is removed from office, he/she shall be barred from attending meetings of any branch of the GSG for no less than one full legislative session, unless invited by said branch.
		\item At the conclusion of the impeachment proceedings, the assembly, with a majority vote, may decide to make the committee report pertaining the impeachment investigation outcome public as is, or amend it.
	\end{bylaws-number}
	\item If, through the course of investigation, the Governance Committee finds that the impeachment proceedings are dependent on pending legal matters, the situation shall be reported to the Assembly. The Assembly, by a majority vote of its total present and voting membership, shall then decide whether or not to continue the impeachment proceedings.
	\item If, through the course of investigation, the impeached party shall decide to resign, the resignation must be submitted to Governance Committee. Upon receiving the resignation, the committee may decide one of the following:
	\begin{enumerate}[i]
		\item Accept and conclude the investigation
		\item Reject and continue the investigation
		\item Accept but continue the investigation
	\end{enumerate}
	\begin{bylaws-number}
		\item In any of these cases, the committee shall communicate their decision in writing with the Assembly prior to the next scheduled meeting.
		\item If committee shall decide to accept but continue the investigation, they shall submit the outcome of their investigation to the Assembly in the next scheduled meeting and the Assembly may accept the outcome with a simple majority vote and conclude the impeachment proceedings.
		\item If the impeached party shall decide to resign during the investigation and the committee accept the resignation, they shall be bared attending meetings of any branch of the GSG indefinitely, unless invited by said branch.
	\end{bylaws-number}
\end{bylaws-number}


\section{Conflict of Interest}
Any impeached member of the Governance Committee must recuse himself/herself from any participation in an impeachment investigation.