\chapter{Elections Code}

\section{Purpose}

The purpose of the GSG Elections Code is to regulate the elections process for all GSG offices with regard to candidacy, campaign procedures, polling, balloting, and any and all aspects of elections processes; and to create a fair environment for all elections proceedings.

\section{Elections Committee Qualifications and Tasks}
\begin{bylaws-number}
  \item The GSG Faculty Advisor or the Coordinator for Graduate Student Life, or their designee, shall serve as the official advisor to the Elections Committee.
  \begin{bylaws-number}
    \item No member of the committee may be a candidate for any Executive Office.
    \item Any member of the Elections Committee seeking elected office in the Assembly shall be required to recuse him/herself from any hearings, deliberations, or rulings of the committee pertaining to their specific race.
    \item No member of the committee may campaign for any Executive Office candidate.
    \item Should the Vice President for Community Development run for any Executive position, they shall:
    \begin{bylaws-number}
      \item Step down from the Elections Committee and a chair pro tem shall be elected from the current committee membership for the duration of the annual election.
      \item Continue to coordinate advertising and communicating the occurrence of the annual election as part of his or her Vice Presidential responsibilities.
    \end{bylaws-number}
  \end{bylaws-number}
\end{bylaws-number}

\section{Scope and Authority}
\begin{bylaws-number}
  \item Full authority on all GSG election matters shall lie with the Elections Committee, subject to the lines of authority set forth in the Constitution. The committee is empowered to establish additional policies and procedures governing elections.
  \item All candidates must be in full compliance with this Code and any other rules outlined by the committee from the commencement of any election proceedings until the election results have been certified, publicly announced, and any appeals have been heard and considered by the Elections Committee and, if necessary, the Governance Committee. Violations of this Code and/or committee rulings, as determined by the committee, shall be grounds for a candidate’s disqualification.
  \item The responsibility for understanding all election rules and regulations lies with the candidates, but every effort should be made by the Elections Committee to make the election policies and procedures clear.
\end{bylaws-number}

\section{Manner of Elections}
\begin{bylaws-number}
  \item A general election to fill all Executive and Assembly positions shall be held annually. All voting in GSG general elections shall be web-based.
  \item Names shall appear on the ballot in alphabetical order.
  \item No party names, logos, or slogans shall appear on the ballot.
  \item Each graduate student may cast only one vote per race.
\end{bylaws-number}

\section{Declaration of Winners}
\begin{bylaws-number}
  \item The winner in a given race shall be the candidate receiving a plurality of votes cast; no candidate may be elected to any office without receiving a minimum of two votes.
  \item The committee advisor, in conjunction with the Elections Committee, shall certify the election results. Such certification will include a letter from the Coordinator for Graduate Student Life, addressed to the Assembly, verifying the results are accurate.
  \item Announcement of the results of the election shall be made immediately after verification.
\end{bylaws-number}

\section{Runoff Elections and Additional Election Procedures}
\begin{bylaws-number}
  \item Ties in any race shall be decided by a runoff election.
  \item If the Elections Committee determines that a runoff election is necessary, it shall be held no later than two weeks following the announcement of the election results.
  \item In the case of a tie in any runoff election, the winner(s) shall be determined by the Assembly in a vote by secret ballot.
  \item All campaign rules apply to runoff elections.
  \item At the conclusion of the general election, should any Executive position remain vacant due to a lack of candidates or any other circumstance, it shall be filled according to the procedures outlined in Article 6.17.
\end{bylaws-number}

\section{Candidate Eligibility Requirements}
Only currently enrolled graduate students may run for office in the GSG.
 
\section{Candidate Nomination Procedures}
The Elections Committee shall establish procedures for the nomination of candidates and make a public announcement concerning nominations and the timeline for the elections. Nominations shall be open for at least 4 weeks for general elections and 2 weeks for midterm elections.

\section{Executive Elections Documents}
\begin{bylaws-number}
  \item The Elections Committee shall prepare and make public appropriate elections documents for all candidates for Executive Office; these documents shall be made available no later the start of the nomination period.
  \item The election documents shall include:
  \begin{enumerate}
    \item Nomination instructions,
    \item A Candidate Information Letter,
    \item An Election Calendar,
    \item A Candidate Agreement Statement,
    \item A detailed job description of all Executive Offices,
    \item A copy of this Code, 
    \item If available, a copy of the campus rules governing the posting of campaign materials, and
    \item Any other form deemed appropriate by the Elections Committee.
  \end{enumerate}
\end{bylaws-number}

\section{Candidacy}
\begin{bylaws-number}
  \item Nominees will become candidates after meeting the published requirements of the election procedure. The committee shall announce the time and date for the end of the nominations process at least one week prior to closing nominations.
  \item The committee shall make public the names of all candidates no later than five business days after the close of nominations.
  \item In the event that the regular nominations process produces no candidate for a given position, the committee may determine and announce procedures for late nominations.
\end{bylaws-number}

\section{Campaign Regulations and Procedures}
\begin{bylaws-number}
  \item All candidates are responsible for their actions and the actions of their surrogates. Thus, surrogates must comply fully with this Code and any rules adopted by the Elections Committee that apply to the election proceedings.
  \item Every candidate shall be required to submit a written statement for publication under guidelines determined by the committee.
  \item Campaign materials shall meet current university policy including all restrictions on posting flyers.
  \item Candidates shall be responsible for removing all of their campaign and publicity materials posted during the campaigning period within two business days of the end of the election period.
  \item The committee shall announce and host at least one Executive Meet and Greet that shall be open to the entire University community.
\end{bylaws-number}

\section{Campaign Finances}
\begin{bylaws-number}
  \item The Elections Committee shall establish campaign expenditure limits equal to or less than \$100 per candidate, and shall announce these limits in the Candidate Information Letter.
  \item Goods and services donated in kind shall be counted toward the maximum set by the committee. Calculations of value shall be made using current, fair market prices.
  \item Each candidate shall maintain financial records of all money and donations received and paid out for campaigning, which shall be updated and available to the committee within two business days of being requested by the committee. All candidates for Executive Office shall be required to submit to the committee, within 5 business days of the close of the elections period, their final financial records, an archive of which shall be duly maintained by the Vice President for Public Relations and made public at the request of any graduate student.
\end{bylaws-number}

\section{Elections Code Violations/Complaints}
\begin{bylaws-number}
  \item Any graduate student currently enrolled at the University, including any member of the Elections Committee, may file complaints regarding violations of this Code and/or any rules established by the committee regarding the election process.
  \item All complaints must be filed in writing to the committee. The committee shall communicate with any candidate accused of violations of the Code within two business days following receipt of the complaint.
  \item Complaints may be resolved informally at the discretion of the committee. If either party is dissatisfied with such a resolution, they may request an Election Code Violation Hearing, which must be granted.
\end{bylaws-number}

\section{Election Code Violation Hearings}
\begin{bylaws-number}
  \item An Election Code Violation Hearing shall be held within two business days of being requested by any party to a complaint. The specified time and place of the Hearing shall be made public by the Elections Committee, and all Hearings shall be open to the entire University community.
  \item The individual filing the complaint must present their case at the Hearing and demonstrate that a violation occurred by a preponderance of the evidence If the individual fails to appear at the Hearing, the committee may dismiss the charges.
  \item All candidates and other individuals who are charged with violating this Code shall be given the opportunity to defend themselves at the Hearing.
  \item The Chair of the Elections Committee shall preside over the Hearing.
  \item After hearing both cases presented, the committee shall deliberate in closed session and make public their determination within one business day.
\end{bylaws-number}

\section{Violation Penalties}
\begin{bylaws-number}
  \item The Elections Committee shall impose an appropriate penalty upon any candidate or campaign if a majority of the committee determines that the candidate, or one of the candidate’s surrogates, has violated this Code or any other rules governing the election process.
  \item If a candidate or their surrogate is found to have violated this Code or any other rules governing the election process, the committee shall vote on an appropriate penalty. The committee may choose to issue a written warning, or disqualify the candidate from the election.
\end{bylaws-number}

\section{Special Circumstances}
In the event of religious holidays, University-wide emergencies, and/or any other extenuating circumstances that may interfere with the election process, the Elections Committee may change any time constraints regarding elections.

\section{Mid-term Elections Procedures – Executives}
In the event that any Executive Office becomes vacant during a legislative term for any reason, or following a general election in which an Executive Office is left unfilled:
\begin{bylaws-number}
  \item The Executive Committee, or its designee, shall circulate an announcement of the election to take place at the next Assembly meeting occurring at least two weeks after the announcement of the vacancy. A call for nominations for the vacant office should be announced using all available and appropriate channels.
  \item Nominations for the vacant office shall be formally taken from the floor during New Business at said meeting.
  \item Nominees not present at said meeting may accept their nomination through written notice submitted to the Vice President for Legislative Affairs prior to the meeting.
  \item After nominations are closed by the Presiding Officer of the Assembly, each candidate shall be given not more than three minutes to present their platform, and not more than three minutes to answer questions from Representatives.
  \item After all the candidates have been heard, they will be asked to wait outside of the meeting room so that the Assembly can have a free and open discussion of the candidates, not to exceed ten minutes, prior to voting.
  \item The Assembly shall vote by secret ballot.
  \item A plurality of votes cast, not counting abstentions, shall be required for a candidate’s election.
  \item Election results shall be immediately tabulated and announced by a group consisting of the Presiding Officer of the Assembly, the GSG Faculty Advisor or their designee and/or the Coordinator for Graduate Student Life.
\end{bylaws-number}

\section{Mid-term Elections Procedures – Representatives}
In the event of a vacancy in any Assembly seat during a legislative term for any reason:
\begin{bylaws-number}
  \item The Elections Committee shall take all necessary and appropriate steps to notify students within the program in question of any vacancies.
  \item Any interested graduate student currently enrolled in the program in question may nominate themselves for the vacant seat by contacting the Elections Committee.
  \item Upon receipt of such a nomination, the Elections Committee shall take all necessary and appropriate steps to inform students in the program in question that nominations for the vacant position have been received, and that a mid-term election will be held to fill the seat.
  \item Once the committee is reasonably confident that students in the program have been notified, the online elections system will be open for additional nominations, and the committee shall announce a closing date for the nominations period. The nominations period shall last a minimum of seven business days.
  \item Following the close of the nominations period, the committee shall announce the official ballot and open a special election period, which will be a minimum of seven business days in length. Students may vote for candidates within their program through the online elections system.
  \item Following the close of the polls, the candidate receiving a plurality of votes cast shall be declared the winner, subject the provisions of Article 6.5.A.
  \begin{bylaws-number}
   \item In the case that no candidate receives a minimum of two votes, the committee may extend the special election period.
   \item Should the special election period result in a tie between two or more candidates, the committee shall announce a runoff election, following the procedures outlined in Article 6.6.
   \item In the case of an uncontested race, or a race in which the number of vacancies exceeds the number of candidates, the Elections Committee may call an election once a candidate receives the minimum number of votes.
  \end{bylaws-number}
  \item The committee advisor, in conjunction with the Elections Committee, shall certify the election results. Such certification will include a letter from the Coordinator for Graduate Student Life, addressed to the Assembly, verifying the results are accurate.
  \item Under ordinary circumstances, the results of the election shall be made public immediately after verification by appropriate parties; should a runoff election be required, results shall be made public following final certification of a winner by the Elections Committee and the committee advisor.
  \item In the event that the outcome of the special election and a subsequent runoff election remains a tie, the winner shall be decided by a majority vote of the Assembly at its next regularly scheduled meeting. Prior to said vote, each candidate is entitled to address the Assembly for no more than two minutes.
  \item After the candidates have been heard, they will be asked to wait outside of the meeting room so that the Assembly can have a free and open discussion of the candidates, not to exceed five minutes, prior to voting.
  \item The election will be conducted by secret ballot.
  \item A plurality of votes cast, not counting abstentions, will be required for a candidate’s election.
  \item Election results shall be immediately tabulated and announced by a group consisting of the Presiding Officer of the Assembly, the GSG Faculty Advisor or their designee, and the committee advisor.
\end{bylaws-number}

\section{Special General Elections}
A special general election to fill all current vacancies in the GSG, which shall comply with the rules and regulations stated above governing a general election, may be called at any time by submitting a resolution to that effect to the Assembly.

\section{Elections Appeals}
\begin{bylaws-number}
	\item Upon receipt of an appeal from a candidate, the Governance Committee shall rule on the decision of the Elections Committee in question no later than forty-eight hours from receipt of said appeal. The committee may uphold, overturn, or amend a decision of the Elections Committee as it deems appropriate.
	\item Decisions on all appeals of Elections Committee decisions shall be by a majority vote of the committee.
	\item The committee shall issue an opinion reflecting the reasoning behind the ruling of the majority in each decision. Committee members voting in the minority shall be entitled to submit their own joint or individual minority opinions.
\end{bylaws-number}