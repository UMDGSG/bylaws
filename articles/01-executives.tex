\chapter{Executives}

\section{Executive Committee}
\begin{bylaws-number}
  \item In addition to the Executive Committee Offices defined in the Constitution, the Graduate Student Government (GSG) shall elect the following officers:
  \begin{bylaws-number}
    \item Vice President for Graduate Student Affairs
    \item Vice President for Community Development
    \item Vice President for Academic Affairs
    \item Vice President for Public Relations
    \item Vice President for Diversity and Inclusion
    \item Vice President for Government Affairs
  \end{bylaws-number}
  \item All the members of the Executive Committee, GSG Executive Staff and GSG Hourly Staff shall be enrolled graduate students.
  \item The Executive Committee shall carry out all policies of the Assembly (Assembly), as well as use its own discretion to act in the best interests of graduate students.
  \item Members of the Executive Committee (Executives) shall be ex-officio, non-voting members of the Assembly, with the exception of the Vice President for Legislative Affairs, who shall exercise a tie-breaking vote and all other rights and privileges accorded to the Presiding Officer by the parliamentary authority. Executives shall be allowed, after proper recognition, to request a Main Motion of Subsidiary Motion be made by any Representative or by a specific Representative, but they may not make such a motion themselves. Executives shall be allowed to make Incidental Motions, with the exception of a Motion to Suspend the Rules, which is the exclusive privilege of the Assembly.
  \item Each Executive shall present a report to the Assembly at regularly scheduled meetings or as directed by the Assembly. Additionally, a written copy of the report should be submitted for inclusion in the minutes.
  \item Executive Committee attendance at monthly Assembly meetings is mandatory.
  \item Executives shall perform such duties as the President may assign them in the administration of the Executive Committee, with a majority approval of the Executive Committee.
  \item Executives shall carry out all duties and responsibilities as directed by the Assembly. Specific duties and responsibilities of these officers shall be specified below.
  \item Any Executive who seeks to resign from his/her elected position shall give notice to the Assembly and, when possible, assist with the transition of his/her position to a new officer. If at any point during his/her term an Executive, for any reason, is found to no longer be a currently enrolled graduate student or if the student graduated in the Spring Semester and it is past June 30th, that position shall immediately become vacant, and shall be filled according to the procedures set forth in Article 6.17.
  \item Any Executive or Program Representative (Representative) who is removed from office by an impeachment proceeding shall be barred from holding any office within the GSG for no less than one complete legislative session.
  \item The incoming President shall be responsible for selecting, in consultation with the incoming Executive Committee, a Director of Operations. The incoming Executive Committee must approve the Director of Operations by a majority vote.
  \item Presidential Disability and Succession
  \begin{bylaws-number}
    \item If, for any reason, the office of the President becomes vacant, or the President is unable to discharge his/her duties, the Vice President for Legislative Affairs shall serve as President, as described in Article 3.3.A of the Constitution.
    \item Whenever a majority of the Executive Committee shall transmit to the Presiding Officer of the Assembly their written declaration that the President is unable to discharge the powers and duties of his/her office, the Vice President for Legislative Affairs shall immediately assume the powers and duties of the office as President pro tem. Should the office of Vice President for Legislative Affairs be vacant, the Executive Committee shall appoint, by majority vote, one of their number to serve as President pro tem. Thereafter, when the President transmits to the Presiding Officer of the Assembly his/her written declaration that no further inability exists, he/she shall resume the powers and duties of the office.
    \item In the event that a President pro tem serves for a period of more than one month, he/she shall be financially compensated from the GSG treasury for his/her service at a rate determined by the Assembly.
  \end{bylaws-number}
  \item All members of the GSG Executive Committee and GSG Executive Staff shall participate in a formal onboarding process the academic semester prior to the beginning of their terms. This will include the current/outgoing GSG Executive Committee and GSG Executive Staff as well as the newly elected/incoming GSG Executive Committee and GSG Executive staff.
  \begin{bylaws-number}
  	\item This onboarding shall take place whenever there is an incoming Executive Committee or staff member.
  	\item This onboarding process shall be scheduled by and led by the current/outgoing Executive Committee members and staff.
  	\item The onboarding program should include, when possible, participation from the Graduate School Dean and Assistant Dean, the GSG Advisor(s), the Graduate Student Life Coordinator, the University Senate Office, the Director of Stamp Student Union, The Student Organization Resource Center Business Manager, the Graduate
  	Student Legal Aid Director, the Vice President for Student Affairs and the University President’s Office.
  	\item The time commitment for newly elected GSG Executive Committee members and appointed GSG Executive staff will be for no more than 8 hours prior to the commencement of the official elected term.
  	\item Each exiting Executive Committee member shall provide a memo in conjunction with the standard transitional materials (handbook, files, etc.) that guides the new Executive Committee member on the most needed and relevant aspects of their role.
  	\begin{bylaws-number}
  	  \item Suggested topics to be included in this memo are shared
  	  governance, ethics, organizational and communication structure of the university, GSG
  	  governing documents and procedures, budget policies and processes, and ongoing
  	  projects.
  	\end{bylaws-number}
  	\item The onboarding process for those elected in the general elections must be completed prior to the end of the spring academic semester.
  \end{bylaws-number}
  \item All representatives of the GSG on campus wide committees or councils, whether ad-hoc or permanent, appointed by the GSG President or any other executive, shall be confirmed by the assembly with a majority vote.
  \item The GSG President shall be jointly titled as 1) The Graduate Student Government President; and 2) The Graduate Student Body President.
\end{bylaws-number}
\section{Duties and Responsibilities of the President}
\begin{bylaws-number}
  \item The President shall be the Chief Executive Officer of the organization and the Chair of the Executive Committee. The duties and responsibilities of the President shall be to:
  \begin{bylaws-number}
    \item Call for a referendum of the graduate student body when directed to do so by a two-thirds vote of the Assembly’s total present and voting membership.
    \item Nominate delegates to those campus organizations that provide the GSG a seat, or other University bodies, when asked to do so by the appropriate authorities. The terms of appointments shall coincide with the term of the appointing officer. All delegates are strongly encouraged, though not required, to attend meetings of the Assembly and the Executive Committee. All delegates shall present reports to the Assembly at least once each semester. The President has the authority to remove delegates by a majority vote of the Executive Committee. This shall include, but not be limited to, the following positions:
    \begin{enumerate}[i]
      \item GSG Presidential Advisor on Senate-Related Issues
      \item GSG Representative to the Senate Educational Affairs Committee
      \item GSG Representative to the Senate Campus Affairs Committee
      \item GSG Representative to the Senate Student Affairs Committee
      \item GSG Representative to the Stamp Student Union Advisory Board
      \item GSG Representative to the Campus Transportation Parking Advisory Committee
      \item GSG Representative to the Campus Recreation Advisory Board
      \item GSG Representative to the Athletic Council
      \item GSG Representative to the Campus Student Technology Fee Advisory Committee
      \item GSG Representative to the Information Technology Council
    \end{enumerate}
    \item The President or their designee(s), as members of the Committee for the Review of Student Fees (CRSF), shall be required to present the proposed fee increases and any pertinit information to the Assembly for the upcoming fiscal year. The presentation date shall be determined by the president prior to the CRSF meeting for the purpose of transparency and awareness of the appropriation of student fees for the upcoming fiscal year.
    \item Oversee the administration of the GSG Executive Offices including the hiring and termination of the GSG Chief-of-Staff and any other office staff, and the termination of the GSG Director of Operations.
    \item Serve as a member of the Graduate Council.
    \item Serve as an ex-officio, non-voting member of the University Senate.
    \item Serve as the GSG Representative to the Committee for the Review of Student Fees.
    \item Serve as the GSG Representative to the Kirwan Faculty Award Committee.
    \item Serve on the committee which appoints all members of the Student Honor Council, in accordance with the procedures detailed in that body’s Bylaws.
  \end{bylaws-number}
  \item The President shall be empowered to issue Executive Orders, which shall be formal declarations of any official decision or action and which shall not exceed the scope of authority granted to the office.
\end{bylaws-number}

\section{Duties and Responsibilities of the Vice President for Legislative Affairs}
The duties and responsibilities of the Vice President for Legislative Affairs shall be to:
\begin{bylaws-number}
  \item Serve as the chair, with voting privileges, of the Rules Committee.
  \item Oversee preparation and dissemination of minutes from Assembly meetings.
  \item Prepare, in consultation with the Executive Committee, a schedule of Assembly meetings for each legislative session, to be called the GSG Assembly Meeting Calendar.
  \item Cast a tie-breaking vote in the Assembly.
  \item Maintain a roster of all current Representatives and a list of all active program codes.
  \item Serve as a voting member of the Legislative Action Committee
  \item Ensure that an accurate list of all attendees at meetings of the Assembly is recorded and preserved.
  \item Upon learning and verifying that a Representative is no longer enrolled in credits toward a graduate degree, or in the case of students graduating in spring until the end of fiscal year of the year of their graduation, declare that position vacant and notify the Elections Committee of the vacancy.
\end{bylaws-number}

\section{Duties and Responsibilities of the Vice President for Financial Affairs}
The duties and responsibilities of the Vice President for Financial Affairs shall be to:
\begin{bylaws-number}
  \item Oversee the financial affairs of the GSG in accordance with the Financial Policies and Procedures specified in these Bylaws (\chaptername \ref{finance}).
  \item Serve as the chair, with voting privileges, of the Budget and Finance Committee.
\end{bylaws-number}

\section{Duties and Responsibilities of the Vice President for Graduate Student Affairs}
The duties and responsibilities of the Vice President for Graduate Student Affairs shall be to:
\begin{bylaws-number}
  \item Monitor actions taken by the University that affect graduate students and to report these actions to the Assembly.
  \item Serve as a voting member of the Graduate Student Affairs Committee.
  \item Serve as the liaison for issues relating to graduate student health, housing, and transportation.
\end{bylaws-number}

\section{Duties and Responsibilities of the Vice President for Community Development}
The duties and responsibilities of the Vice President for Community Development shall be to:
\begin{bylaws-number}
  \item Encourage graduate students to participate in all governing bodies of the University and the GSG, as well as in graduate student organizations.
  \item Organize and oversee activities, as directed by the Social and Sport and Executive Committees, and as approved by the Assembly, dedicated to increasing the participation of graduate students in the life of the campus and in the GSG.
  \item Serve as the liaison between the GSG and all registered graduate student organizations on behalf of the Executive Committee, creating and developing a working relationships with said organizations.
  \item Maintain regular contact with registered graduate student organizations regarding GSG business, funding opportunities, and possibilities for partnership with the GSG as directed by the Executive Committee.
  \item Serve as a voting member of the Social and Sport Committee.
\end{bylaws-number}

\section{Duties and Responsibilities of the Vice President for Academic Affairs}
The duties and responsibilities of the Vice President for Academic Affairs shall be to:
\begin{bylaws-number}
  \item Investigate University and Graduate School policies which pertain to the academic and professional development of graduate students.
  \item Review and develop GSG policy and programs related to the academic and professional development of graduate students.
  \item Serve as a voting member of the University Senate Educational Affairs Committee, and represent the GSG to the Senate on all matters of academic policy.
  \item Serve as a member of the Provost’s Student Advisory Council.
  \item Serve as one of the two GSG members of the Graduate Council.
  \item Serve as a voting member of the GSG Academic Affairs Committee.
  \item Serve as chair of the Graduate Research Appreciation Day (GRAD) Conference Planning Committee.
  \item Serve as the co-chair of the Provost’s Student Advisory Council (ProSAC) and appoint members to the council on behalf of the GSG, as governed by the memorandum of understanding between the GSG and the Office of the Provost.
  \item Serve as a member of Graduate Council on behalf of the GSG or designate someone to if they are unable
  \item Serve as a member of Graduate Council on behalf of the GSG or designate someone to if they are unable
\end{bylaws-number}

\section{Duties and Responsibilities of the Vice President for Public Relations}
The duties and responsibilities of the Vice President for Public Relations shall be to:
\begin{bylaws-number}
  \item Chair the Elections Committee and oversee elections.
  \item Conduct outreach to departments and programs with unfilled seats, with special effort to recruit members for programs without any Assembly representation.
  \item Communicate the work of the GSG with the campus community and beyond, including creating and disseminating press releases, newsletter(s), and other communiqués.
  \item Work closely with the VPCD in engaging student participation from student organizations.
  \item Collaborate with the chairs of GSG committees to publicize their work, with a particular emphasis on the Social \& Sports Committee and Graduate Student Life to substantially increase attendance of their respective events.
  \item Serve as the GSG liaison with media, including the Diamondback.
  \item Draft a report at the end of each year including metrics by which the worth of the position shall be judged, including, but not limited to, open representative slots filled during year, number of representatives elected in the annual election and number of graduate students voting in the annual election.
\end{bylaws-number}

\section{Duties and Responsibilities of Vice President for Diversity and Inclusion}
The duties and responsibilities of the Vice President for Diversity and Inclusion shall be to:
\begin{bylaws-number}
	\item Chair the GSG Diversity Committee.
	\item Advise GSG committees on diversity issues affecting other GSG activities.
	\item Act as liaison between the University of Maryland and the Graduate Student Government on all diversity-related issues.
	\item Serve as GSG’s representative on university diversity-related committees, or designate someone else to if they are unable.
\end{bylaws-number}

\section{Duties and Responsibilities of the Director of Operations}
The duties and responsibilities of the Director of Operations shall be to:
\begin{bylaws-number}
  \item Serve as chief administrator of the organization and its resources including the maintenance and operation of the GSG Offices.
  \item Administer, on behalf of the Executive Committee and Assembly, the financial accounts of the organization under the direction of the President and the Vice President for Financial Affairs.
  \item Maintain documentation of all GSG long-lived property and insure all property has appropriate labeling. Track consumables and gift items purchased by GSG. Present an inventory report to the Executive Committee at least once a semester. No GSG property may be removed from its documented location, unless mobility is a part of the item’s application (e.g. laptops, ipads, etc.).
  \item Administer the Event Funding Request (EFR) process in accordance with the directives of the Assembly and the Budget and Finance Committee.
  \item Serve as an Ex-Officio, non-voting member of the Executive Committee.
  \item Attend all meetings of the Assembly and assist the Vice President for Legislative Affairs in the administration of that body.
  \item Carry out such other duties and responsibilities as the Executive Committee and Assembly may see fit to require in the administration of the organization.
\end{bylaws-number}

\section{Duties and Responsibilities of the Chief of Staff}
\begin{bylaws-number}
	\item The Chief of Staff shall oversee and coordinate the functioning of standing and ad hoc committees of the GSG, oversee the staffing of standing and ad hoc committees as described in Article 5, and, in conjunction with the President, oversee and coordinate the placement of graduate representatives on University committees, and serve as a liaison to graduate student members of the University Senate committees.
\end{bylaws-number}

\section{Duties and Responsibilities of Other Presidential Appointees}
\begin{bylaws-number}
  \item Presidential Appointees are strongly encouraged, though not required, to attend meetings of the Assembly and the Executive Committee.
  \item The President or a Representative may require the submission of a written report from Appointees on any activities pertaining to their position.
\end{bylaws-number}

\section{Attendance Standards}
\begin{bylaws-number}
  \item Executives shall attend all meetings of the Assembly. Should an absence be unavoidable, Executives shall communicate their report, in written or electronic form, to the Presiding Officer of the Assembly, who shall ensure it is presented at the meeting.
  \item Missing three regularly scheduled Assembly meetings shall automatically result in a charge of Neglect. At the first regularly scheduled Assembly meeting following an Executive’s third absence, the Assembly shall vote on whether to initiate impeachment proceedings against said Executive, according to the procedures set forth in Article 3. Before the Assembly’s vote, the Executive shall be permitted an opportunity to explain his/her absences, and answer questions from the Assembly. Should the motion to impeach not carry, the Assembly has the right to consider impeachment charges against the Executive following any subsequent absence from a regularly scheduled Assembly meeting.
\end{bylaws-number}