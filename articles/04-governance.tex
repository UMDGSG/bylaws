\chapter{Governance Committee}

\section{Composition}
\begin{bylaws-number}
  \item The committee shall be composed of seven members:
  \begin{bylaws-number}
    \item One Executive, chosen by the Executive Committee;
    \item Three Representatives, chosen by the Assembly; and
    \item Three other currently enrolled graduate students who are not elected members of the GSG, appointed by the President.
The President may not serve on the Governance Committee.
  \end{bylaws-number}
  \item Members of the committee shall be selected by the conclusion of the first regular meeting of the Assembly in September.
  \item The Assembly shall select members of the committee by nomination from the floor. In the event that there are more than three nominees, the Assembly shall select from among them using the procedures outlined in Article 6.17.
  \item The term of the committee shall be concurrent with the legislative session.
  \item In the event of a vacancy, the position shall be filled at the earliest possible time according to the appropriate procedure detailed above.
\end{bylaws-number}

\section{General Procedures}
\begin{bylaws-number}
  \item The committee’s first order of business at the beginning of each session shall be to meet for the purposes of selecting a chair and reviewing and agreeing upon a set of standing rules. The committee shall also select a vice-chair who shall preside over meetings in the absence of the chair.
  \item Quorum shall be set at five members.
  \item If the committee meets to hear an appeal or to oversee impeachment proceedings, the Coordinator for Graduate Student Life or the GSG Faculty Advisor, or his or her designee, shall be present as an observer.
  \item The committee may meet in closed session if considering business which involves personnel matters normally kept confidential by the University.
  \item Minutes of all meetings shall be kept.
\end{bylaws-number}

\section{Constitutional Matters}
\begin{bylaws-number}
  \item Upon receipt of a valid appeal of the constitutionality of any action taken by the GSG, the Governance Committee shall convene to consider the question. The committee shall rule on the constitutionality of the action in question no later than five school days after the receipt of said appeal.
  \item In all questions of constitutionality, the decision of the committee shall be by majority vote.
  \item The committee shall issue an opinion reflecting the reasoning behind the ruling of the majority in each decision. Committee members voting in the minority shall be entitled to submit their own joint or individual minority opinions.
  \item The Governance Committee shall be the highest authority in the GSG concerning all matters of interpretation of the GSG Constitution, Bylaws, legislation, Executive Orders, and all other statutes.
  \item The committee shall have the authority to overturn any legislation, impeachment decision or disciplinary action, or Executive Order that is found to be in conflict with the GSG Bylaws or Constitution.
  \item The committee cannot overturn a constitutional amendment approved by the Assembly for consideration in an upcoming referendum by the entire graduate student body, except in cases where the procedures by which the amendment was placed before the Assembly, or by which the Assembly disposed of the amendment, violate the procedures outlined in these Bylaws or in the Assembly’s parliamentary authority. The committee’s authority must be exercised prior to the inclusion of the amendment in a referendum placed before the entire graduate student body.
\end{bylaws-number}

\section{Impeachment}
\begin{bylaws-number}
  \item Upon the initiation of impeachment proceedings, the question of impeachment shall be forwarded to the Governance Committee.
  \item The Governance Committee shall be charged with the unbiased investigation of all matters related to the impeachment charges.
  \begin{bylaws-number}
    \item The Governance Committee shall conclude whether the impeached party should be formally indicted on the charges as brought forth by the Assembly, and shall formulate, for the Assembly, a recommendation to either proceed with or terminate impeachment proceedings.
    \item The Governance Committee shall be charged with the duty of disclosing to the impeached party the timelines and processes involved in any potential impeachment hearing of the committee and the Assembly. This shall include the distribution of these impeachment protocols to the impeached party.
    \item The investigation shall consist of no less than an interview with the impeached party with representation or a witness pending notification of the committee, an interview with the impeaching party, and the gathering of supporting documentation and other such evidence to be presented to the Assembly. If any of the above mentioned parties are unavailable for interview, that information can be taken into account and should be presented to the Assembly.
    \item The committee shall present a copy of its findings and recommendations to the accused party and the impeaching party no less than twenty-four hours before an anticipated presentation of said findings before the Assembly.
  \end{bylaws-number}
  \item Impeachment proceedings shall commence at the next regularly scheduled meeting of the Assembly. By majority vote, the committee may appeal to the Assembly for an extension of the investigation period, not to exceed one additional month or the duration until the next Assembly meeting, whichever is greater. If such an appeal is met favorably, impeachment proceedings shall commence at the next regularly scheduled meeting of the Assembly following the end of the extension period. Such an extension should be avoided unless judged to be absolutely necessary to the work of the committee.
  \begin{bylaws-number}
    \item At said meeting of the Assembly, a designee of the committee shall present the findings of the court’s investigation, with a recommendation to proceed with or terminate impeachment proceedings. This recommendation shall be considered a question for debate and shall be decided by a simple majority of the Assembly’s present and voting membership. The committee’s recommendation shall take the following form:
“After considering evidence in the form of corroborated personal communications with the involved parties, and any other additional evidence made available to the committee, and after considerable deliberation, it is the decision of this committee to formally recommend that the Assembly [insert recommendation].”
    \begin{enumerate}[i]
      \item Dismiss: Dismiss the [insert member’s official title] of all impeachment charges.
      \item Impeach: Indict the [insert member’s official title] for impeachment under the charge(s) of [insert formal charge(s)].
      \item Dismiss the charge(s) of [insert dismissed charge(s)], and to formally indict the [insert member’s official title] for impeachment under the charge(s) of [insert formal charge(s)].
    \end{enumerate}
    \item Members of the committee wishing to dissent from the majority opinion may present to the Assembly a written dissenting opinion. This dissenting opinion may be forwarded to the Assembly anonymously through the office of the Chair. Anonymous opinions must contain the number of committee members that are signatory.
    \item If the Assembly elects to proceed with the impeachment, the Assembly shall immediately enter executive session and shall conduct an impeachment trial.
    \item The impeachment trial shall be presided over by the Assembly’s Presiding Officer. The individual/s bringing impeachment charges shall serve as the prosecution and the impeached party shall be charged with mounting his/her defense. The impeached party has the right to be aided in his/her defense by counsel. Counsel can be an attorney or not, but must be a member of the Assembly, unless the Assembly, by majority vote, agrees to permit counsel who is not a member to act in this capacity.
    \item At the conclusion of the trial, the Assembly shall engage in debate and vote upon the question of removing the impeached party from office.
    \item The Assembly, by a three-fourths vote of its total present and voting membership, shall remove the impeached party from his/her position(s). If a party is removed from office, he/she shall be barred from attending meetings of any branch of the GSG for one year, unless invited by said branch.
  \end{bylaws-number}
  \item If, through the course of investigation, the Governance Committee finds that the impeachment proceedings are dependent on pending legal matters, the situation shall be reported to the Assembly. The Assembly, by a majority vote of its total present and voting membership, shall then decide whether or not to continue the impeachment proceedings.
\end{bylaws-number}

\section{Elections Appeals}
\begin{bylaws-number}
  \item Upon receipt of an appeal from a candidate, the Governance Committee shall rule on the decision of the Elections Committee in question no later than forty-eight hours from receipt of said appeal. The committee may uphold, overturn, or amend a decision of the Elections Committee as it deems appropriate.
  \item Decisions on all appeals of Elections Committee decisions shall be by a majority vote of the committee.
  \item The committee shall issue an opinion reflecting the reasoning behind the ruling of the majority in each decision. Committee members voting in the minority shall be entitled to submit their own joint or individual minority opinions.
\end{bylaws-number}