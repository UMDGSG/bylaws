\chapter{Governance Committee}

\section{Composition}
\begin{bylaws-number}
  \item The committee shall be composed of seven members:
  \begin{bylaws-number}
    \item One Executive, chosen by the Executive Committee;
    \item Three Representatives, chosen by the Assembly; and
    \item Three other currently enrolled graduate students who are not elected members of the GSG, appointed by the President.
The President may not serve on the Governance Committee.
  \end{bylaws-number}
  \item Members of the committee shall be selected by the conclusion of the first regular meeting of the Assembly in September.
  \item The Assembly shall select members of the committee by nomination from the floor. In the event that there are more than three nominees, the Assembly shall select from among them using the procedures outlined in Article 6.17.
  \item The term of the committee shall be concurrent with the legislative session.
  \item In the event of a vacancy, the position shall be filled at the earliest possible time according to the appropriate procedure detailed above.
\end{bylaws-number}

\section{General Procedures}
\begin{bylaws-number}
  \item The committee’s first order of business at the beginning of each session shall be to meet for the purposes of selecting a chair and reviewing and agreeing upon a set of standing rules. The committee shall also select a vice-chair who shall preside over meetings in the absence of the chair.
  \item Quorum shall be set at five members.
  \item If the committee meets to hear an appeal or to oversee impeachment proceedings, the Coordinator for Graduate Student Life or the GSG Faculty Advisor, or his or her designee, shall be present as an observer.
  \item The committee may meet in closed session if considering business which involves personnel matters normally kept confidential by the University.
  \item Minutes of all meetings shall be kept.
\end{bylaws-number}

\section{Constitutional Matters}
\begin{bylaws-number}
  \item Upon receipt of a valid appeal of the constitutionality of any action taken by the GSG, the Governance Committee shall convene to consider the question. The committee shall rule on the constitutionality of the action in question no later than five school days after the receipt of said appeal.
  \item In all questions of constitutionality, the decision of the committee shall be by majority vote.
  \item The committee shall issue an opinion reflecting the reasoning behind the ruling of the majority in each decision. Committee members voting in the minority shall be entitled to submit their own joint or individual minority opinions.
  \item The Governance Committee shall be the highest authority in the GSG concerning all matters of interpretation of the GSG Constitution, Bylaws, legislation, Executive Orders, and all other statutes.
  \item The committee shall have the authority to overturn any legislation, impeachment decision or disciplinary action, or Executive Order that is found to be in conflict with the GSG Bylaws or Constitution.
  \item The committee cannot overturn a constitutional amendment approved by the Assembly for consideration in an upcoming referendum by the entire graduate student body, except in cases where the procedures by which the amendment was placed before the Assembly, or by which the Assembly disposed of the amendment, violate the procedures outlined in these Bylaws or in the Assembly’s parliamentary authority. The committee’s authority must be exercised prior to the inclusion of the amendment in a referendum placed before the entire graduate student body.
\end{bylaws-number}

\section{Elections Appeals}
\begin{bylaws-number}
  \item Upon receipt of an appeal from a candidate, the Governance Committee shall rule on the decision of the Elections Committee in question no later than forty-eight hours from receipt of said appeal. The committee may uphold, overturn, or amend a decision of the Elections Committee as it deems appropriate.
  \item Decisions on all appeals of Elections Committee decisions shall be by a majority vote of the committee.
  \item The committee shall issue an opinion reflecting the reasoning behind the ruling of the majority in each decision. Committee members voting in the minority shall be entitled to submit their own joint or individual minority opinions.
\end{bylaws-number}